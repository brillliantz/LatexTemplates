%%%%%%%%%%%%%%%%%%%%%%%%%%%%%%%%%%%%%%%%%%%%%%%%%%%%%%%%%%%
%     _______ _    _  _____       
%    |__   __| |  | |/ ____|
%       | |  | |__| | (___  
%       | |  |  __  |\___ \ 
%       | |  | |  | |____) |
%       |_|  |_|  |_|_____/
%
%     _               ____  
%    | |        /\   |  _ \ 
%    | |       /  \  | |_) |
%    | |      / /\ \ |  _ < 
%    | |____ / ____ \| |_) |
%    |______/_/    \_\____/ 
%
% _______ ______ __  __ _____  _            _______ ______ 
%|__   __|  ____|  \/  |  __ \| |        /\|__   __|  ____|
%   | |  | |__  | \  / | |__) | |       /  \  | |  | |__   
%   | |  |  __| | |\/| |  ___/| |      / /\ \ | |  |  __|  
%   | |  | |____| |  | | |    | |____ / ____ \| |  | |____ 
%   |_|  |______|_|  |_|_|    |______/_/    \_\_|  |______|
%
%%%%%%%%%%%%%%%%%%%%%%%%%%%%%%%%%%%%%%%%%%%%%%%%%%%%%%%%%%%
%%%%%%%%%%%%%%%%%%%%%%%%%%%%%%%%%%%%%%%%%%%%%%%%%%%%%%%%%%%
%%%%% DONT CHANGE ANYTHING BEFORE THE "TITLE" SECTION.%%%%%
%%%%%%%%%%%%%%%%%%%%%%%%%%%%%%%%%%%%%%%%%%%%%%%%%%%%%%%%%%%
%%%%%%%%%%%%%%%%%%%%%%%%%%%%%%%%%%%%%%%%%%%%%%%%%%%%%%%%%%%
\documentclass{article} % Especially this!
% _____        _____ _  __          _____ ______  _____ 
%|  __ \ /\   / ____| |/ /    /\   / ____|  ____|/ ____|
%| |__) /  \ | |    | ' /    /  \ | |  __| |__  | (___  
%|  ___/ /\ \| |    |  <    / /\ \| | |_ |  __|  \___ \ 
%| |  / ____ \ |____| . \  / ____ \ |__| | |____ ____) |
%|_| /_/    \_\_____|_|\_\/_/    \_\_____|______|_____/ 
%%%%%%%%%%%%%%%%%%%%%%%%%%%%%%%%%%%%%%%%%%%%%%%%%%%%%%%%

\usepackage[english]{babel}
\usepackage[utf8]{inputenc}
\usepackage[margin=1.0in]{geometry}
\usepackage{amsmath}
\usepackage{amsthm}
\usepackage{amsfonts}
\usepackage{amssymb}
\usepackage[usenames,dvipsnames]{xcolor}
\usepackage{graphicx}
\usepackage[siunitx]{circuitikz}
\usepackage{tikz}
\usepackage[colorinlistoftodos, color=orange!50]{todonotes}
\usepackage{hyperref}
\usepackage[numbers, square]{natbib}
\usepackage{fancybox}
\usepackage{epsfig}
\usepackage{soul}
\usepackage[framemethod=tikz]{mdframed}
\usepackage[shortlabels]{enumitem}
\usepackage[version=4]{mhchem}
%------------------- Personalized Packages
\usepackage{fontspec}
\usepackage{lipsum}
\usepackage{subfigure}
\usepackage{ctex}
\usepackage{booktabs}
\usepackage{multirow}
%\usepackage[crop=off]{auto-pst-pdf}% With an onstalled Perl always crop=on!
%\ifpdf
%\usepackage{forest}
%\else
%\usepackage{pst-barcode}
%\fi
%------------------- minted package coding environment
\usepackage[english]{babel}
\usepackage{minted}
\usemintedstyle{vs}
%------------------- write2tex
\usepackage{xunicode}
\usepackage{xltxtra}
%\usepackage{polyglossia}
\usepackage{color}
\usepackage{array}
\usepackage{supertabular}
\usepackage{hhline}



%%%%%%%%%%%%%%%%%%%%%%%%%%%%%%%%%%%%%%%%%%%%%%%%%%%%%%%


%   _____ _    _  _____ _______ ____  __  __ 
%  / ____| |  | |/ ____|__   __/ __ \|  \/  |
% | |    | |  | | (___    | | | |  | | \  / |
% | |    | |  | |\___ \   | | | |  | | |\/| |
% | |____| |__| |____) |  | | | |__| | |  | |
%  \_____|\____/|_____/   |_|  \____/|_|  |_|
%%%%%%%%%%%%%%%%%%%%%%%%%%%%%%%%%%%%%%%%%%%%%%%%
%  _____ ____  __  __ __  __          _   _ _____   _____ 
% / ____/ __ \|  \/  |  \/  |   /\   | \ | |  __ \ / ____|
%| |   | |  | | \  / | \  / |  /  \  |  \| | |  | | (___  
%| |   | |  | | |\/| | |\/| | / /\ \ | . ` | |  | |\___ \ 
%| |___| |__| | |  | | |  | |/ ____ \| |\  | |__| |____) |
% \_____\____/|_|  |_|_|  |_/_/    \_\_| \_|_____/|_____/ 
%%%%%%%%%%%%%%%%%%%%%%%%%%%%%%%%%%%%%%%%%%%%%%%%%%%%%%%%%%

% SYNTAX FOR NEW COMMANDS:
%\newcommand{\new}{Old command or text}

% EXAMPLE:

\newcommand{\blah}{blah blah blah \dots}

%%%%%%%%%%%%%%%%%%%%%%%%%%%%%%%%%%%%%%%%%%%%%%%%%%%%%%%%%
%  _______ ______          _____ _    _ ______ _____    %
% |__   __|  ____|   /\   / ____| |  | |  ____|  __ \   %
%    | |  | |__     /  \ | |    | |__| | |__  | |__) |  %
%    | |  |  __|   / /\ \| |    |  __  |  __| |  _  /   %
%    | |  | |____ / ____ \ |____| |  | | |____| | \ \   %
%    |_|  |______/_/    \_\_____|_|  |_|______|_|  \_\  %
%%%%%%%%%%%%%%%%%%%%%%%%%%%%%%%%%%%%%%%%%%%%%%%%%%%%%%%%%
%                                                       %
%           COMMANDS                SUMMARY             %
% \clarity{points}{comment} >>> "Clarity of Writing"    %
% \other{points}{comment}   >>> "Other"                 %
% \spelling{comment}        >>> "Spelling"              %
% \units{comment}           >>> "Units"                 %
% \english{comment}         >>> "Syntax and Grammer"    %
% \source{comment}          >>> "Sources"               %
% \concept{comment}         >>> "Concept"               %
% \missing{comment}         >>> "Missing Content"       %
% \maths{comment}           >>> "Math"                  %
% \terms{comment}           >>> "Science Terms"         %
%                                                       %
%%%%%%%%%%%%%%%%%%%%%%%%%%%%%%%%%%%%%%%%%%%%%%%%%%%%%%%%%
\setlength{\marginparwidth}{2.4cm}


% NEW COUNTERS
\newcounter{points}
\setcounter{points}{100}
\newcounter{spelling}
\newcounter{english}
\newcounter{units}
\newcounter{other}
\newcounter{source}
\newcounter{concept}
\newcounter{missing}
\newcounter{math}
\newcounter{terms}
\newcounter{clarity}

% COMMANDS

\definecolor{myblue}{rgb}{0.668, 0.805, 0.929}
\newcommand{\hlb}[2][myblue]{ {\sethlcolor{#1} \hl{#2}} }

\newcommand{\clarity}[2]{[color=CornflowerBlue!50]{CLARITY of WRITING(#1) #2}\addtocounter{points}{#1}
\addtocounter{clarity}{#1}}

\newcommand{\other}[2]{\todo{OTHER(#1) #2} \addtocounter{points}{#1} \addtocounter{other}{#1}}

\newcommand{\spelling}{\todo[color=CornflowerBlue!50]{SPELLING (-1)} \addtocounter{points}{-1}
\addtocounter{spelling}{-1}}
\newcommand{\units}{\todo{UNITS (-1)} \addtocounter{points}{-1}
\addtocounter{units}{-1}}

\newcommand{\english}{\todo[color=CornflowerBlue!50]{SYNTAX and GRAMMAR (-1)} \addtocounter{points}{-1}
\addtocounter{english}{-1}}

\newcommand{\source}{\todo{SOURCE(S) (-2)} \addtocounter{points}{-2}
\addtocounter{source}{-2}}
\newcommand{\concept}{\todo{CONCEPT (-2)} \addtocounter{points}{-2}
\addtocounter{concept}{-2}}

\newcommand{\missing}[2]{\todo{MISSING CONTENT (#1) #2} \addtocounter{points}{#1}
\addtocounter{missing}{#1}}

\newcommand{\maths}{\todo{MATH (-1)} \addtocounter{points}{-1}
\addtocounter{math}{-1}}
\newcommand{\terms}{\todo[color=CornflowerBlue!50]{SCIENCE TERMS (-1)} \addtocounter{points}{-1}
\addtocounter{terms}{-1}}


\newcommand{\summary}[1]{
\begin{mdframed}[nobreak=true]
\begin{minipage}{\textwidth}
\vspace{0.5cm}
\begin{center}
\Large{Grade Summary} \hrule 
\end{center} \vspace{0.5cm}
General Comments: #1

\vspace{0.5cm}
Possible Points \dotfill 100 \\
Points Lost (Science Terms) \dotfill \theterms \\
Points Lost (Syntax and Grammar) \dotfill \theenglish \\
Points Lost (Spelling) \dotfill \thespelling \\
Points Lost (Units) \dotfill \theunits \\
Points Lost (Math) \dotfill \themath \\
Points Lost (Sources) \dotfill \thesource \\
Points Lost (Concept) \dotfill \theconcept \\
Points Lost (Missing Content) \dotfill \themissing \\
Points Lost (Clarity of Writing) \dotfill \theclarity \\
Other \dotfill \theother \\[0.5cm]
\begin{center}
\large{\textbf{Grade:} \fbox{\thepoints}}
\end{center}
\end{minipage}
\end{mdframed}}


%\newcommand\textsubscript[1]{\ensuremath{{}_{\text{#1}}}}
\makeatletter
\newcommand\arraybslash{\let\\\@arraycr}
\makeatother
\setlength\tabcolsep{1mm}
\renewcommand\arraystretch{1.3}


%#########################################################

%To use symbols for footnotes
\renewcommand*{\thefootnote}{\fnsymbol{footnote}}
%To change footnotes back to numbers uncomment the following line
%\renewcommand*{\thefootnote}{\arabic{footnote}}

% Enable this command to adjust line spacing for inline math equations.
% \everymath{\displaystyle}

% _______ _____ _______ _      ______ 
%|__   __|_   _|__   __| |    |  ____|
%   | |    | |    | |  | |    | |__   
%   | |    | |    | |  | |    |  __|  
%   | |   _| |_   | |  | |____| |____ 
%   |_|  |_____|  |_|  |______|______|
%%%%%%%%%%%%%%%%%%%%%%%%%%%%%%%%%%%%%%%

\title{
    \normalfont \normalsize 
    \textsc
    {\kaishu 南京大学 \quad 匡亚明学院 \\ 
        近代物理实验(二)} \\
    [10pt] 
    \rule{\linewidth}{0.5pt} \\[6pt] 
    \huge 实验11.4 \quad 矢量网络分析仪测量微波材料的介电常数和磁导率 \\
    \rule{\linewidth}{2pt}  \\[10pt]
}
\author{刘冰楠(141210016)}
\date{\normalsize 2016年09月29日}

\begin{document}%%%%%%===================================

\maketitle
\noindent
实验日期 \dotfill 2016年09月22日 \\
共同实验者 \dotfill 陈鲲, 陈谦 \\
指导老师 \dotfill 陈建军 \\

%%%%%%%%%%%%%%%%%%%%%%%%%%%%%%%%%%%%%%%

%             ______      ____  
%            |  ____/\   / __ \ 
%            | |__ /  \ | |  | |
%            |  __/ /\ \| |  | |
%            | | / ____ \ |__| |
%            |_|/_/    \_\___\_\
%%%%%%%%%%%%%%%%%%%%%%%%%%%%%%%%%%%%%%%%

%
% Ctrl + / to comment out a group of lines.
%
%
% LIST MORE COMMON COMMMANDS
% LIST USEFUL WEBSITES FOR TABLES, ETC
% WHAT TO DO WHEN YOUR CODE WONT COMPILE
% OVERLEAF SHORTCUTS
%



%%%%%%%%%%%%%%%%%%%%%%%%%%%%%%%%%%%%%%%


% _               ____  
%| |        /\   |  _ \ 
%| |       /  \  | |_) |
%| |      / /\ \ |  _ < 
%| |____ / ____ \| |_) |
%|______/_/    \_\____/ 
%%%%%%%%%%%%%%%%%%%%%%%%
%  _____ _______       _____ _______ _____ 
% / ____|__   __|/\   |  __ \__   __/ ____|
%| (___    | |  /  \  | |__) | | | | (___  
% \___ \   | | / /\ \ |  _  /  | |  \___ \ 
% ____) |  | |/ ____ \| | \ \  | |  ____) |
%|_____/   |_/_/    \_\_|  \_\ |_| |_____/ 
%%%%%%%%%%%%%%%%%%%%%%%%%%%%%%%%%%%%%%%%%%%
% _    _ ______ _____  ______ 
%| |  | |  ____|  __ \|  ____|
%| |__| | |__  | |__) | |__   
%|  __  |  __| |  _  /|  __|  
%| |  | | |____| | \ \| |____ 
%|_|  |_|______|_|  \_\______|
%%%%%%%%%%%%%%%%%%%%%%%%%%%%%%
\section{摘要}
%#TODO 增加定量描述
本实验通过了解到X射线的产生、特点和应用;理解X射线管产生连续X射线谱和特征X射线谱的基本原理,了解D8X射线衍射仪的基本原理和使用方法,通过分析软件对测量样品进行定性的物相分析。

\begin{figure}[htbp]
    \centering
    \includegraphics[width=0.7\textwidth]{resources/test_sample.png}
    \caption{test\_sample}
    \label{fig:test_sample}
\end{figure}
\lipsum[1]


%这是一个CTEX的utf-8编码例子,{\kaishu 这里是楷体显示},{\songti 这里是宋体显示},{\heiti 这里是黑体显示},{\fangsong 这里是仿宋显示}。

\newpage

%实验的意义,为什么要做。背景,历史
\section{引言}



\section{实验目的}



\section{实验原理}

\section {实验仪器}


%#################################################################
% _____  _____   ____   _____ ______ _____  _    _ _____  ______ 
%|  __ \|  __ \ / __ \ / ____|  ____|  __ \| |  | |  __ \|  ____|
%| |__) | |__) | |  | | |    | |__  | |  | | |  | | |__) | |__   
%|  ___/|  _  /| |  | | |    |  __| | |  | | |  | |  _  /|  __|  
%| |    | | \ \| |__| | |____| |____| |__| | |__| | | \ \| |____ 
%|_|    |_|  \_\\____/ \_____|______|_____/ \____/|_|  \_\______|
%%%%%%%%%%%%%%%%%%%%%%%%%%%%%%%%%%%%%%%%%%%%%%%%%%%%%%%%%%%%%%%%%%%
\section {实验步骤}
%%%%%%%%%%%%%%%%%%%%%%%
% FOR A NUMBERED LIST
% \begin{enumerate}
% \item Your_Item
% \end{enumerate}
%%%%%%%%%%%%%%%%%%%%%%%



%##################################################################
% _____       _______       
%|  __ \   /\|__   __|/\    
%| |  | | /  \  | |  /  \   
%| |  | |/ /\ \ | | / /\ \  
%| |__| / ____ \| |/ ____ \ 
%|_____/_/    \_\_/_/    \_\
%%%%%%%%%%%%%%%%%%%%%%%%%%%%%%
\section {数据测量}
%%%%%%%%%%%%%%%%%%%%%%%%%%%%%%
% TO IMPORT AN IMAGE
% UPLOAD IT FIRST (HIT THE PROJECT BUTTON TO SHOW FILES)
% KEEP THE NAME SHORT WITH NO SPACES!
% TYPE THE FOLLOWING WITH THE NAME OF YOUR FILE
% DON'T INCLUDE THE FILE EXTENSION
% \includegraphics[width=\textwidth]{name_of_file}
% \textwidth makes the picture the width of the paragraphs
%%%%%%%%%%%%%%%%%%%%%%%%%%%%%%
% TO CREATE A FIGURE WITH A NUMBER AND CAPTION
% \begin{figure}
% \includegraphics[width=\textwidth]{image}
% \caption{Your Caption Goes Here}
% \label{your_label}
% \end{figure}
% REFER TO YOUR FIGURE LATER WITH
% \ref{your_label}
% LABELS NEED TO BE ONE WORD
%%%%%%%%%%%%%%%%%%%%%%%%%%%%%


%任何测量都要带着不确定度。所有数据要成表。总结成果,找出问题,提出改进。一方面求得的测量结果假装不知道其真值,求不确定度;另一方面与真值比较看是否在合理范围,所有不合理的要分析原因。回答开头提出的目的
\section{误差分析和校正}



%#############################
% _____ _____  _____  _____ _    _  _____ _____ 
%|  __ \_   _|/ ____|/ ____| |  | |/ ____/ ____|
%| |  | || | | (___ | |    | |  | | (___| (___  
%| |  | || |  \___ \| |    | |  | |\___ \\___ \ 
%| |__| || |_ ____) | |____| |__| |____) |___) |
%|_____/_____|_____/ \_____|\____/|_____/_____/ 
%%%%%%%%%%%%%%%%%%%%%%%%%%%%%%%%%%%%%%%%%%%%%%%%
\section {思考题与进一步讨论}
%%%%%%%%%%%%%%%%%%%%%%%%%%%%%%%%%%%%%%%%%%%%%%%%
\subsection{思考题}

\textbf{1、根据波导传输理论推导s参量与介电常数的关系式。}\\


\textbf{2、本实验测得材料的介电常数其主要误差来源是什么?}\\



%  _____  ____  _    _ _____   _____ ______  _____ 
% / ____|/ __ \| |  | |  __ \ / ____|  ____|/ ____|
%| (___ | |  | | |  | | |__) | |    | |__  | (___  
% \___ \| |  | | |  | |  _  /| |    |  __|  \___ \ 
% ____) | |__| | |__| | | \ \| |____| |____ ____) |
%|_____/ \____/ \____/|_|  \_\\_____|______|_____/ 
%%%%%%%%%%%%%%%%%%%%%%%%%%%%%%%%%%%%%%%%%%%%%%%%%%%%
% USE NOCITE TO ADD SOURCES TO THE BIBLIOGRAPHY WITHOUT SPECIFICALLY CITING THEM IN THE DOCUMENT

\nocite{modernphyexp}

%%%%%%%%%%%%%%%%%%%%%%%%%%%%%%%%%%%%%%%%%%%%%%%%%%%%%%
% BIBLIOGRAPHY: %

% Make sure your class *.bib file is uploaded to this project by clicking the project button > add files. Change 'sample' below to the name of your file without the .bib extension.
%%%%%%%%%%%%%%%%%%%%%%%%%%%%%%%%%%%%%%%%%%%%%%%%%%

\bibliographystyle{plainnat}
\bibliography{mybib}

% UNCOMMENT THE TWO LINES ABOVE TO ENABLE BIBLIOGRAPHY
%%%%%%%%%%%%%%%%%%%%%%%%%%%%%%%%%%%%%%%%%%%%%%%%%%

\section*{附录: 所用到的计算机程序代码}
\subsection*{数据读取}
\inputminted[
frame=lines,
framesep=2mm,
baselinestretch=1.2,
%bgcolor=darkgray,
fontsize=\footnotesize,
linenos
]{python}{Resources/test_sample.py}

\end{document} % NOTHING AFTER THIS LINE IS PART OF THE DOCUMENT

%\begin{figure}[htbp]
%    \centering
%    \includegraphics[width=0.7\textwidth]{resources/system.png}
%    \caption{实验系统示意框图}
%    \label{fig:system}
%\end{figure}

% ______ ______ ______ ______ ______ ______ ______ 
%|______|______|______|______|______|______|______|
%                  _____  ____  
%                 / ____|/ __ \ 
%                | |  __  |  | |
%                | | |_ | |  | |
%                | |__| | |__| |
%                 \_____|\____/ 
% 
% _    _  ____  _____  _   _ ______ _______ _____ _ 
%| |  | |/ __ \|  __ \| \ | |  ____|__   __/ ____| |
%| |__| | |  | | |__) |  \| | |__     | | | (___ | |
%|  __  | |  | |  _  /| . ` |  __|    | |  \___ \| |
%| |  | | |__| | | \ \| |\  | |____   | |  ____) |_|
%|_|  |_|\____/|_|  \_\_| \_|______|  |_| |_____/(_)
%
% ______ ______ ______ ______ ______ ______ ______ 
%|______|______|______|______|______|______|______|